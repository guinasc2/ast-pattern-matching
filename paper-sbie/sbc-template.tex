\documentclass[12pt]{article}

\usepackage{sbc-template}
\usepackage{graphicx,url}
\usepackage[utf8]{inputenc}
\usepackage[brazil]{babel}
% \usepackage[latin1]{inputenc} 

% \usepackage{hyperref}
\usepackage{amsmath,amsthm,amssymb}
\usepackage{proof,tikz}
\usetikzlibrary{automata, positioning, arrows}

\usepackage[Algoritmo]{algorithm}
\usepackage{algorithmicx}
\usepackage{algpseudocode}
\usepackage{listings}

\sloppy

\title{---}

\author{Anônimo}


\address{Departamento de Computação -- Universidade Federal de Ouro Preto
  (UFOP)\\
  35400-000 -- Ouro Preto -- MG -- Brazil
}

\begin{document} 

\maketitle

\begin{abstract}
  -----
\end{abstract}
     
\begin{resumo}
  -----
\end{resumo}


\section{Introdução}\label{Introducao}

Com o aumento de cursos destinados a programação de computadores, muitos 
professores passaram a usar juízes automáticos (ou corretores automáticos de 
código-fonte) para auxiliar na correção de exercícios e fornecer 
\textit{feedback} mais imediato aos alunos. Imagine que você é um professor de
uma disciplina de programação e está avaliando uma questão cujo enunciado é o
seguinte:

``Você foi contratado pelo Ministério do Meio Ambiente para avaliar a meta de 
reflorestamento das regiões brasileiras e vai implementar um programa para 
ajudá-lo em suas análises. Para facilitar a coleta de dados, cada estado é 
dividido em microrregiões. Você recebe periodicamente um vetor de valores 
inteiros indicando a quantidade mínima de árvores nativas plantadas para cada 
estado, representando a meta de cada estado, e uma matriz de valores inteiros 
que mostra a quantidade de árvores plantadas em cada estado em cada microrregião,
as linhas da matriz representam as microrregiões e as colunas os estados. 
As entradas do vetor e da matriz são feitas por meio das funções inputVetor e 
inputMatriz, respectivamente (definidas no livro texto da disciplina).

Seu programa calcula o total de árvores plantadas pelos estados e avalia se eles 
cumpriram com a meta (quantidade de árvores plantadas é igual ou superior à meta 
do estado), imprimindo no terminal os estados que não conseguiram cumprir a meta 
(os números dos estados começam de 1, embora os índices comecem de 0, então, índice 
0 representa o estado 1, índice 1 representa o estado 2, e assim por diante). 
A relação entre o vetor e a matriz se dá pelos índices dos elementos do vetor e os 
índices de coluna da matriz.''

E, ao abrir a solução submetida por um aluno, se depara com esse código:

\begin{lstlisting}[language=Python, breaklines]
  def inputVetor():
    entrada = input("Informe as metas dos estados: ")
    return list(map(int, entrada.split(',')))

  def inputMatriz():
    entrada = input("Informe o plantio de arvores: ")
    linhas = entrada.split(';')
    matriz = [list(map(int, linha.split(','))) for linha in linhas]
    return matriz

  def main():
    print("Ministerio do Meio Ambiente")
    metas = inputVetor()
    plantio = inputMatriz()
    
    num_estados = len(metas)
    totais_plantio = [sum(linha[i] for linha in plantio) for i in range(num_estados)]
    
    for i in range(num_estados):
      if totais_plantio[i] < metas[i]:
        print(f"Estado {i+1}, meta = {metas[i]}, plantio = {totais_plantio[i]}")

  if __name__ == "__main__":
    main()
\end{lstlisting}

Que embora correto e gere a resposta esperada, usa recursos da linguagem Python 
que ignoram os objetivos de aprendizagem pretendidos ou não foram apresentados 
na disciplina, como o uso das funções \texttt{map} e \texttt{sum}, 
\textit{list comprehension} e o uso do atributo \texttt{\_\_name\_\_}.
Para atender a necessidade de não apenas validar a resposta, mas também validar
o código submetido pelo aluno, propomos uma bibliteca para análise de código 
multi-linguagem baseada em Parsing Expression Grammars (PEGs)\cite{ford2004-peg} 
na linguagem Haskell, projetada para detectar o uso de construções avançadas ou não 
autorizadas pelos alunos, visando auxiliar professores a impor limites pedagógicos.

% TODO: explicar o que é uma PEG de forma bem simplificada

\section{Biblioteca: escolher um nome}

% A biblioteca utiliza Árvores Abstratas de Síntaxe (ASTs) para realizar casamento
% de padrões. A forma da AST é determinada pela PEG utilizada para fazer o parsing
% de determinado arquivo.
A biblioteca permite ao usuário definir PEGs e padrões, e oferece operações 
para processar outros tipos de arquivos usando essas definições. A PEG é usada
para realizar a análise sintática do arquivo processado, gerando uma Árvore
Abstrata de Síntaxe (AST) do arquivo. Com a AST gerada, é possível realizar
casamento de padrões, capturar subárvores e reescrever trechos da árvore,
fazendo, assim, alterações no texto original.
% A seguir, um exemplo de uma PEG para expressões envolvendo operações de soma,
% multiplicação e parênteses com números e variáveis:

% \begin{lstlisting}[label=ex:peg-expr,caption=PEG para expressões,frame=tb]
% E   <- T ("+" T)*
% T   <- F ("*" F)*
% F   <- Num / Var / "(" E ")"
% Num <- [0-9]+
% Var <- [a-zA-Z] [a-zA-Z_0-9]* 
% \end{lstlisting}

% E um padrão para casar com expressões que envolvem multiplicações:

% \begin{lstlisting}[label=ex:pat-expr-mul,caption=Padrão para expressões,frame=tb]
% pattern mult : T := #e1:F ("*" #e2:F)*
% \end{lstlisting}

Para tentar restringir o uso de construções indevidas, foi escrita uma PEG que
aceita um \textit{subset} de Python, permitindo apenas o que foi apresentado ao
longo da disciplina de Programação de Computadores, como: definições e chamadas 
de funções, os comandos para decisão (\texttt{if}, \texttt{elif}, \texttt{else}), 
para repetição (\texttt{while}, \texttt{for}), as duas formas de importação
(\texttt{import X} e \texttt{from X import Y}), os operadores de atribuição,
lógicos e aritméticos, vetores e matrizes.
Porém, apenas isso não é suficiente para eliminar totalmente todas as construções
não autorizadas, como usar uma função definida em alguma biblioteca para resolver
o problema, em vez de desenvolver a própria solução. Na Seção~\ref{sec:estudos-caso}
são apresentados dois estudos de caso: um trata de uma forma de resolver o problema 
apresentado anteriormente e o segundo, dada a capacidade da biblioteca para 
reescrever a AST de um programa, trata de sugestões para reescrita do código, 
com o intuito de mostrar ao aluno formas de simplificar e melhorar a estrutura 
algorítmica do programa.


\section{Estudos de caso} \label{sec:estudos-caso}

Para avaliar e mostrar as capacidades da biblioteca, apresentamos a seguir dois
estudos de caso: um envolve uma sugestão de reescrita do código, enquanto o outro 
trata da verificação da presença de construções não autorizados na solução de um 
aluno.

\subsection{Validação da resposta do aluno}

Considere uma questão que pede ao aluno para implementar um programa que calcule
o fatorial de um número inteiro \(n\) digitado pelo usuário. Definimos como 
\(n!\) (\(n\) fatorial) a multiplicação sucessiva de \(n\) por seus antecessores 
até chegar em 1. 
A equação do fatorial pode ser definida como \(n! = n \times n-1 \times n-2 
\times \ldots 3 \times 2 \times 1\). A solução esperada é que o aluno utilize
um laço de repetição, como o \textit{while}, para implementar a multiplicação
sucessiva dos números, como no código apresentado a seguir:

\begin{lstlisting}[language=Python]
  n = int(input("Digite um numero: "))
  fatorial = 1
  contador = n
  while (contador >= 1):
    fatorial = fatorial * contador
    contador = contador - 1
  print(f"{n}! = {fatorial}")
\end{lstlisting}

Porém, dentro da biblioteca \texttt{math} de Python, existe a função 
\texttt{factorial}, que dado um número inteiro \(n\), retorna o resultado de \(n!\).
Por isso, alguns alunos acabam importando a biblioteca e usando a função pronta,
contornando o objetivo do exercício, que é praticar o laço de repetição.
Dessa forma, o padrão apresentado a seguir é capaz de identificar a presença de
uma chamada para a função.

% TODO: não tô gostando de usar o listing, pq não consigo colocar uma label nele, mas não sei o que usar no lugar
%   o ambiente de matemática tb não fica legal, mas pelo menos aceita o epsilon
\[
  \begin{array}{ll}
    pattern factorial\_call : & function\_call := (identifier := "math.factorial") @space "(" @space \#v2:expr_list? ")" \epsilon \\
    pattern space : & space := " "*
    
  \end{array}
\]

Onde:
\begin{itemize}
  \item A palavra \textit{pattern} inicializa a declaração de um padrão.
  \item \textit{factorial\_call} é o identificador do padrão.
  \item \textit{function\_call} indica o tipo do padrão, i.e., com quais comandos 
    ele vai casar.
  \item \textit{(identifier := "math.factorial")} significa que o nome da função 
    deve ser \textit{math.factorial}.
  \item \textit{@space} faz referência ao padrão de nome \textit{space}.
  \item \textit{``(''} indica que deve casar com a string ``(''.
  \item \textit{\#v2:expr\_list?} significa que a lista de argumentos da função
    será armazenada na variável \textit{v2}. O \textit{?} indica que essa lista
    de argumentos é opcional.
  \item \textit{\(\epsilon\)} indica que não deve ter nada em sequência da chamada.
\end{itemize}

Assim, se o padrão casar, significa que o aluno está usando a função 
\textit{factorial} da biblioteca \textit{math} em vez de escrever o laço de
repetição, contornando o objetivo original do exercício.

\subsection{Reescrita de código}

Considere agora o seguinte trecho de código:

\begin{lstlisting}[language=Python]
  if not a:
    print("A condicao 'a' e falsa")
  else:
    print("A condicao 'a' e verdadeira")
\end{lstlisting}

Embora o código não apresente nenhum erro, ele pode ser refatorado, com o intuito 
de melhorar a estrutura e, consequentemente, entendimento do código, ao remover o 
\textit{not} da condição do \texttt{if} e trocar os blocos de comando do 
\texttt{if} e do \texttt{else}, da seguinte forma:

\begin{lstlisting}[language=Python]
  if a:
    print("A condicao 'a' e verdadeira")
  else:
    print("A condicao 'a' e falsa")
\end{lstlisting}

Ao identificar esse tipo de construção no código do aluno, é possível sugerir
uma reescrita ao aluno, explicando o motivo da sugestão e a melhoria que ela
traria ao código. Os padrões apresentados a seguir representam uma forma de
detectar a construção apresentada anteriormente e como fazer a reescrita.

\[
  \begin{array}{ll}
    pattern if\_def : & if\_stmt := (("if" @space @expr ":") \#ifBlock:statement*) @elseBlock \\
    pattern elseBlock : & else\_block := ("else" @space ":") \#elseBlock:statement* \\
  
    pattern subst : & if\_stmt := (("if" @space \#condition:expression ":") \#elseBlock:statement*) @elseBlock2 \\
    pattern elseBlock2 : & else\_block := ("else" @space ":") \#ifBlock:statement* \\
  
    pattern expr    : & expression := @orExpr \epsilon \\
    pattern orExpr  : & or\_expr    := @andExpr \epsilon \\
    pattern andExpr : & and\_expr   := "not" @space \#condition:comparison \\
  
    pattern space : & space := " "*
  \end{array}
\]

Onde:
\begin{itemize}
  \item O padrão \textit{if\_def} casa quando encontrar um if que tem como
    condição uma expressão negada.
  \item O padrão \textit{subst} representa a reescrita que será sugerida ao
    aluno.
\end{itemize}

As variáveis \(\#condition:expression\), \(\#ifBlock:statement*\) e \(\#elseBlock:statement*\) 
no padrão \textit{if\_def} capturam, respectivamente, a expressão na condição do 
\texttt{if}, todo o bloco  de comandos do \texttt{if} e todo o bloco de comandos
do \texttt{else}.
A forma com que essas variáveis aparecem no padrão \textit{subst} indica como a
reescrita será realizada. Nesse padrão, é possível ver que o \textit{not} da condição
não aparece mais, enquanto a posição das variáveis dos blocos foram trocadas.
Assim, é possível usar o que foi capturado pelas variáveis no padrão \textit{if\_def}
e colocar nos locais onde as variáveis aparecem no padrão \textit{subst}.
Por fim, apresentamos a reescrita ao aluno, junto da explicação, e fazemos a 
sugestão para melhoria de sua solução.

\section{Trabalhos relacionados}

Estudantes aprendem melhor vendo representações de um conceito, sejam textuais, 
visuais ou animadas. Os livros didáticos podem oferecer as representações 
textuais e algumas visuais, enquanto os softwares podem fornecer representações 
visuais e animadas. No entanto, apenas observar não é suficiente, os alunos 
devem ser capazes de interagir com o conceito de alguma forma e receber 
\textit{feedback} para verificar sua compreensão do conceito 
\cite{rodger2002-hands-on-visualization}.
Dessa forma, avaliar a corretude da solução do aluno e oferecer sugestões de
melhoria podem contribuir significativamente no aprendizado.

\cite{pelz2012-mecanismo-correcao-automatica} apresenta um mecanismo de correção
automática de programas com um processo de 4 etapas, em que é são feitas as 
verificações da sintaxe, da presença de comandos obrigatórios, da adequação da
estrutura do programa do aluno e dos valores de saída do programa diante de um
conjunto de testes. A partir das observações que fizeram, concluíram que o 
\textit{feedback} sobre a falta de comandos obrigatórios no programa deve ser
utilizado com cuidado, pois ao ser apresentado ao aluno sem nenhuma precaução
é possível que este descubra a estrutura da solução do problema simplesmente
pelas dicas do mecanismo de correção, sem mesmo refletir sobre como solucionar
o problema. Além disso, o mecanismo de correção automática dificilmente seria 
útil para avaliação de exercícios de programação mais complexos do que aqueles 
apresentados nas primeiras semanas de aula, já que neste caso a quantidade de 
variáveis e a complexidade das estruturas dos programas inviabilizam a definição 
dos esquemas de correção da maneira como estão propostos neste artigo.

\cite{moreira2009-ambiente-ensino} cita diversos benefícios para o professor ao
usar um ambiente para ensino de programação com \textit{feedback} automático, 
como: menor esforço, uma vez que conta com o auxílio da ferramenta, melhor 
administração dos estudantes e de suas tarefas, melhor rastreamento individual 
dos estudantes, melhor qualidade de ensino, devido o maior tempo de prática, 
mais tempo para contato com os estudantes. A utilização de ambientes para 
avaliação automática de exercícios tem sido uma prática comum em universidades
e auxiliam no cotidiano de uma disciplina. Uma vez que o aprendizado de
programação é essencial na carreira do estudante de computação, é necessário
que este consiga adquirir os fundamentos básicos, sendo capaz de avaliar, pensar
logicamente e desenvolver seus pŕoprios algoritmos. Por se tratar de uma disciplina
em que a prática em laboratório é fundamental no desenvolvimento destas habilidades,
é desejável que o professor esteja atento à sua evolução. Contudo, devido às 
turmas de programação geralmente serem compostas por muitos alunos e tempo 
reduzido de aulas, nem sempre isso é possível. Por isso, uma forma de avaliar e
fornecer \textit{feedback} aos alunos pode auxiliar no acompanhamento dessas
habilidades.

As principais abordagens para avaliação automática de um exercício de programação
são a análise dinâmica e a análise estática \cite{oliveira2015-avaliacao-automatica-programacao}. 
A análise dinâmica compara o resultado produzido ao executar o programa do aluno 
com um gabarito, chamado de solução esperada e geralmente produzida por um 
programa escrito pelo professor, para avaliar a corretude, funcionalidade e
eficiência da solução em vários casos de teste. Caso aprovados, o exercício 
recebe nota máxima e, caso contrário, nota mínima para cada caso.
A análise estática avalia a escrita do código-fonte, observando itens como 
erros de programação de ordem sintática como erros sintáticos, semânticos,
estrutural e estilo de programação, além de detectar plágio. Assim, ela 
costuma contemplar o processo de construção do programa.

Neste trabalho, foi apresentado um mecanismo para análise estática do programa 
doaluno, com um foco em coibir o uso de estruturas e construções avançadas ou 
não autorizadas pelos alunos, além de fornecer dicas e sugestões para melhorar
a estrutura algorítmica da solução desenvolvida. Utilizando diferentes gramáticas,
é possível restringir os recursos aos que foram ensinados na disciplina, 
disponibilizando mais recursos à medida que o conteúdo avança ou o quando o 
professor julgar adequado. Com o uso dos padrões, é possível garantir que 
determinados comandos e estruturas estão sendo usadas ou não, por exemplo,
para que o aluno use uma estrutura adequada ao problema e não tente burlar o
objetivo da atividade. Além disso, é possível detectar e sugerir melhorias
na estrutura do código, explicando seu motivo e benefícios.

%%%%%%%%%%%%%%%%%%%%%%%%%%%%%%%%%%%%%%%%%%%%%%%%%%%%%%%
%%% A partir daqui é o código do template da SBC

% \section{General Information}

% All full papers and posters (short papers) submitted to some SBC conference,
% including any supporting documents, should be written in English or in
% Portuguese. The format paper should be A4 with single column, 3.5 cm for upper
% margin, 2.5 cm for bottom margin and 3.0 cm for lateral margins, without
% headers or footers. The main font must be Times, 12 point nominal size, with 6
% points of space before each paragraph. Page numbers must be suppressed.

% Full papers must respect the page limits defined by the conference.
% Conferences that publish just abstracts ask for \textbf{one}-page texts.

% \section{First Page} \label{sec:firstpage}

% The first page must display the paper title, the name and address of the
% authors, the abstract in English and ``resumo'' in Portuguese (``resumos'' are
% required only for papers written in Portuguese). The title must be centered
% over the whole page, in 16 point boldface font and with 12 points of space
% before itself. Author names must be centered in 12 point font, bold, all of
% them disposed in the same line, separated by commas and with 12 points of
% space after the title. Addresses must be centered in 12 point font, also with
% 12 points of space after the authors' names. E-mail addresses should be
% written using font Courier New, 10 point nominal size, with 6 points of space
% before and 6 points of space after.

% The abstract and ``resumo'' (if is the case) must be in 12 point Times font,
% indented 0.8cm on both sides. The word \textbf{Abstract} and \textbf{Resumo},
% should be written in boldface and must precede the text.

% \section{CD-ROMs and Printed Proceedings}

% In some conferences, the papers are published on CD-ROM while only the
% abstract is published in the printed Proceedings. In this case, authors are
% invited to prepare two final versions of the paper. One, complete, to be
% published on the CD and the other, containing only the first page, with
% abstract and ``resumo'' (for papers in Portuguese).

% \section{Sections and Paragraphs}

% Section titles must be in boldface, 13pt, flush left. There should be an extra
% 12 pt of space before each title. Section numbering is optional. The first
% paragraph of each section should not be indented, while the first lines of
% subsequent paragraphs should be indented by 1.27 cm.

% \subsection{Subsections}

% The subsection titles must be in boldface, 12pt, flush left.

% \section{Figures and Captions}\label{sec:figs}


% Figure and table captions should be centered if less than one line
% (Figure~\ref{fig:exampleFig1}), otherwise justified and indented by 0.8cm on
% both margins, as shown in Figure~\ref{fig:exampleFig2}. The caption font must
% be Helvetica, 10 point, boldface, with 6 points of space before and after each
% caption.

% \begin{figure}[ht]
% \centering
% \includegraphics[width=.5\textwidth]{fig1.jpg}
% \caption{A typical figure}
% \label{fig:exampleFig1}
% \end{figure}

% \begin{figure}[ht]
% \centering
% \includegraphics[width=.3\textwidth]{fig2.jpg}
% \caption{This figure is an example of a figure caption taking more than one
%   line and justified considering margins mentioned in Section~\ref{sec:figs}.}
% \label{fig:exampleFig2}
% \end{figure}

% In tables, try to avoid the use of colored or shaded backgrounds, and avoid
% thick, doubled, or unnecessary framing lines. When reporting empirical data,
% do not use more decimal digits than warranted by their precision and
% reproducibility. Table caption must be placed before the table (see Table 1)
% and the font used must also be Helvetica, 10 point, boldface, with 6 points of
% space before and after each caption.

% \begin{table}[ht]
% \centering
% \caption{Variables to be considered on the evaluation of interaction
%   techniques}
% \label{tab:exTable1}
% \includegraphics[width=.7\textwidth]{table.jpg}
% \end{table}

% \section{Images}

% All images and illustrations should be in black-and-white, or gray tones,
% excepting for the papers that will be electronically available (on CD-ROMs,
% internet, etc.). The image resolution on paper should be about 600 dpi for
% black-and-white images, and 150-300 dpi for grayscale images.  Do not include
% images with excessive resolution, as they may take hours to print, without any
% visible difference in the result. 

% \section{References}

% Bibliographic references must be unambiguous and uniform.  We recommend giving
% the author names references in brackets, e.g. \cite{knuth:84},
% \cite{boulic:91}, and \cite{smith:99}.

% The references must be listed using 12 point font size, with 6 points of space
% before each reference. The first line of each reference should not be
% indented, while the subsequent should be indented by 0.5 cm.

\bibliographystyle{sbc}
\bibliography{sbc-template}

\end{document}
