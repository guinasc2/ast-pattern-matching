\chapter{Introduction}\label{chap:intro}

Pattern matching is the act of checking a given sequence fo tokens for the presence
of a given pattern. The match usually must be exact: ``either it
will or will not be a match''. It is frequently used to output the locations (if any)
of a pattern within a token sequence, output some component of the matched pattern,
and to substitute the matching pattern with some other token sequence (i.e., search
and replace). Patterns generally have the form of either sequences or tree structures.
Often, patterns sequences are described using regular expressions.


\section{Objectives}\label{sec:objectives}

The main objective of this work is to formalize the semantics of pattern matching
in syntax trees. Specifically, we plan to:
\begin{enumerate}
    \item Define the semantics for generating a parse tree when executing a parsing expression.
    \item Define the semantics for pattern matching on a parse tree.
    \item Prove properties of the defined semantics.
    \item ...
\end{enumerate}

\section{Contributions}\label{sec:contributions}

Our contributions are:
\begin{itemize}
    \item A type system and operational semantics for generating a parse tree.
    \item ...
\end{itemize}

\section{Dissertation Structure}\label{sec:structure}

The rest of this dissertation is structured as follow: Chapter~\ref{chap:background}
covers the necessary background knowledge used in this work, Chapter~\ref{chap:methodology}
presents the pattern matching and generation of parse tree, Chapter~\ref{chap:results}
discusses some case studies using the proposed approach, Chapter~\ref{chap:future-work}
presents the schedule of next steps, and finally Chapter~\ref{chap:conclusion} concludes
this work.
The code for the parsing and pattern match of parse trees can be found on
\url{https://github.com/guinasc2/ast-pattern-matching}.

\cleardoublepage
