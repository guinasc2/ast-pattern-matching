\chapter{Methodology}\label{chap:methodology}

This chapter presents the proposal of the pattern matching algorithm developed
during the work, as well as the  current state of its implementation.
Section~\ref{sec:parse-tress} details
the execution of a PEG for generating a parse tree and how pattern matching works.
Section~\ref{sec:implementation-details} discusses some implementation details.
Section~\ref{sec:methodology-conclusion} concludes the chapter.

\section{Parse trees}\label{sec:parse-tress}

In order to proper define pattern matching over trees, we need
to define parse trees over an arbitrary PEG. We start by defining
tree syntax and a PEG semantics which produces, as a result, such
trees. Next, we present a typing relation which assigns a parsing
expression for a given tree. Such step is necessary to formally
state the equivalence between our proposed tree-producing semantics
and PEG original semantics proposed by Ford~\cite{Ford04}.

Let \(G = (V, \Sigma, R, e_s)\) be an arbitrary PEG, the meta-variable \(a \in \Sigma\) an
arbitrary alphabet symbol, \(A \in V\) a variable and \(e\) a parsing expression.
The following context-free grammar defines the syntax of a parse tree:
\[
   \begin{array}{lcl}
      t & \to & \hat{\epsilon} \, \mid \, \hat{a} \, \mid \, A@t\,
                    \mid \, \langle t_1, t_2 \rangle\,
                    \mid \, L(e,t) \, \mid \, R(e,t) \, \mid \, [\,] \,\mid\,t:t\,
                    \mid \, \eta \\
   \end{array}
\]
Where \(\hat{\epsilon}\) represents that a parsing expression resulted in
success without consuming any symbol of its input, \(\hat{a}\) represents that
the parsing expression consumed the symbol \(a\) from the input, \(A@t\)
represents that the parsing of the rule \(A \leftarrow e \in R\) was succeesful
and $t$ is a tree for $e$. Notation
\(\langle t_1, t_2 \rangle\) represents that a sequence of parsing expressions
succeeded,
\(L \: t\) and \(R \: t\) both represent that a branch of an ordered choice succeeded,
with \(L \: t\) for the left one and \(R \: t\) for the right one,
% \(L \: t\) represents that the first expression in an ordered choice
% succeeded, \(R \: t\) represents that the second expression in an ordered choice
% succeeded,
\([]\) is an empty list of trees for $e$ and \(t_1 : t_2\) denotes a list of
trees in which $t_1$ is a tree for $e$ and $t_2$ is a tree for $e^*$. Finally,
\(\eta\) represents that a not
predicate was successful.

Executing a parsing expressions produces a parsing tree and is defined by an
inductively defined judgment that relates pairs formed by a parsing expression
and an input string to pairs formed by the generated tree and the remaining string.
Notation \((e,s_ps_r) \Downarrow_G (t,s_r)\) denote that parsing expression \(e\)
consumes the prefix \(s_p\) and generates the parse tree \(t\) from the input string
\(s_ps_r\) leaving the suffix \(s_r\). The notation \((e,s) \Downarrow_G \bot\)
denote the fact that \(s\) cannot be parsed by \(e\). We let meta-variable \(r\)
denote an arbitrary parsing result, i.e., either \(r\) is a pair \((t,s_r)\) or
\(\bot\). We say that an expression \(e\) fails if its execution over an input
produces \(\bot\); otherwise, it succeeds. Figure~\ref{fig:peg-tree-semantics}
defines the PEG semantics for tree generation.

\begin{figure}[H]
   \[
      \begin{array}{cc}
         \infer[_{\{Eps\}}]{(\epsilon,s) \Downarrow_G (\hat{\epsilon},s)}{} &
         \infer[_{\{ChrS\}}]{(a,as_r) \Downarrow_G (\hat{a},s_r)}{} \\ \\
         \infer[_{\{ChrF\}}]{(a,bs_r) \Downarrow_G \bot}{a \neq b} &
         \infer[_{\{Var\}}]{(A,s) \Downarrow_G (A@t, r)}
                        {A \leftarrow e \in R & (e,s) \Downarrow_G (t,r)} \\ \\
         \multicolumn{2}{c}{
            \infer[_{\{Cat_{S1}\}}]{(e_1\,e_2,s_{p_1}s_{p_2}s_r) \Downarrow_G (\langle t_1,t_2 \rangle, s_r)}
                                 {(e_1,s_{p_1}s_{p_2}s_r) \Downarrow_G (t_1,s_{p_2}s_r) &
                                 (e_2,s_{p_2}s_r)\Downarrow_G (t_2,s_r)}
         } \\ \\
         \multicolumn{2}{c}{
            \infer[_{\{Cat_{F2}\}}]{(e_1\,e_2,s_ps_r) \Downarrow_G \bot}
                                 { (e_1,s_ps_r) \Downarrow_G (t_1,s_r) &
                                    (e_2,s_r) \Downarrow_G \bot}} \\ \\
         \infer[_{\{Cat_{F1}\}}]{(e_1\,e_2,s)\Downarrow_G \bot}{(e_1,s) \Downarrow_G \bot} &
         \infer[_{\{Alt_{S1}\}}]{(e_1\,/\,e_2,s_p\,s_r) \Downarrow_G (L \, t,s_r)}
                                {(e_1,s_p\,s_r)\Downarrow_G (t,s_r)} \\ \\
         \multicolumn{2}{c}{
            \infer[_{\{Alt_{S2}\}}]{(e_1\,/\,e_2,s_p\,s_r) \Downarrow_G R \, t}
                                  {(e_1,s_p\,s_r)\Downarrow_G \bot &
                                   (e_2,s_p\,s_r)\Downarrow_G t}
         } \\ \\
         \multicolumn{2}{c}{
            \infer[_{\{Star_{rec}\}}]{(e^\star,s_{p_1}s_{p_2}s_r) \Downarrow_G (t_1 : t_2,s_r)}
                                 {(e,s_{p_1}s_{p_2}s_r) \Downarrow_G (t_1,s_{p_2}s_r) &
                                  (e^\star, s_{p_2}s_r) \Downarrow_G (t_2,s_r)}
         } \\ \\
         \infer[_{\{Star_{end}\}}]{(e^\star,s) \Downarrow_G ([\,],s)}
                                    {(e,s) \Downarrow_G \bot} &
         \infer[_{\{Not_F\}}]{(!\,e,s_p\,s_r) \Downarrow_G \bot}
                          {(e,s_p\,s_r) \Downarrow_G (t,s_r)}\\ \\
         \infer[_{\{Not_S\}}]{(!\,e,s) \Downarrow_G (\eta,s)}
         {(e,s) \Downarrow_G \bot}
           &
         \infer[_{\{ChrNil\}}]{(a,\epsilon) \Downarrow_G \bot}{}
      \end{array}
   \]
   \centering
   \caption{Parsing expressions operational semantics that produces a tree.}
   \label{fig:peg-tree-semantics}
\end{figure}

A parse tree is directly related to its underlying parsing expression.
We formalize this idea using a typing relation between grammars, trees and
parsing expressions. Notation $G\vdash t : e$ means that tree $t$ has
type $e$ using as assumption the variables and rules defined by grammar $G$.
Figure~\ref{fig:tree-typing} presents the rules of tree typing relation.

\begin{figure}[H]
  \[
    \begin{array}{cc}
      \infer[_{\{TEps\}}]{G \vdash \hat{\epsilon} : \epsilon}{} &
      \infer[_{\{TChr\}}]{G\vdash \hat{a} : a}{} \\ \\
      \infer[_{\{TVar\}}]{ G \vdash A@t : A}
                      {A \leftarrow e \in R(G) & G\vdash t : e} &
      \infer[_{\{TNot\}}]{G\vdash \eta : ! e}{}\\ \\
      \infer[_{\{TCat\}}]{G\vdash \langle t_1, t_2\rangle : e_1\,e_2}
                      {G\vdash  t_1 : e_1 & G\vdash t_2 : e_2} &
      \infer[_{\{TLeft\}}]{G\vdash L(e_2,t) : e_1 / e_2}
                       {G\vdash t : e_1} \\ \\
      \infer[_{\{TRight\}}]{G\vdash R(e_1,t) : e_1 / e_2}
                       {G\vdash t : e_2} &
      \infer[_{\{TNil\}}]{G \vdash [] : e^*}{}\\ \\
      \multicolumn{2}{c}{
        \infer[_{\{TCons\}}]{G \vdash t_1 : t_2 : e^*}
                         {G\vdash t_1 : e &
                          G \vdash t_2 : e^*}
      }
    \end{array}
  \]
  \centering
  \caption{Typing relation for parse trees.}
  \label{fig:tree-typing}
\end{figure}

INSERIR TEXTO EXPLICANDO CADA REGRA

Next, we need to show that our tree producing semantics is equivalent to
the original semantics for PEGs. Intuitively, our trees are a strutured
representation of the prefix parsed by a grammar. We name the textual
representation of a parse tree as its flattening which is defined
by recursion on the structure of the parse tree (Figure~\cite{fig:flattening-tree}).

\begin{figure}[H]
  \[
    \begin{array}{lcl}
      |\hat{\epsilon}| & = & \epsilon\\
      |\hat{a}| & = & a\\
      |A@t| & = & |t|\\
      |\langle t_1, t_2 \rangle| & = & |t_1|\,|t_2|\\
      |L\:t| & = & |t|\\
      |R\:t| & = & |t|\\
      |[]| & = & \epsilon \\
      |t_1 : t_2| & = & |t_1|\,|t_2|\\
    \end{array}
  \]
  \centering
  \caption{Flattening of a parse tree.}
  \label{fig:flattening-tree}
\end{figure}

EXPLICAR A FLATTENING.

The typing relation specifies what kind of parse trees are possible to be
constructed for a given parsing expression. Using the flatenning function,
we can formally describe the relation between our tree producing semantics
and PEG original semantics.
The next theorem presents this result.

\begin{theorem}[Semantics equivalence]
  Let $G$ be an arbitrary PEG. If $G \vdash t : e$ then $(e, |t|) \Downarrow_G (t, \epsilon)$ and
  $(e, |t|)\Rightarrow_G (|t|, \epsilon)$
\end{theorem}
\begin{proof}
  Induction over the derivation of $G \vdash t : e$ using the correspondent semantics rule in each case.
\end{proof}

In the next section, we use parse trees and its semantics to formally define
a pattern matching algorithm.

\section{Parse tree patterns and its semantics}

We start this section presenting pattern syntax and its typing relation.
Next, we present a subtyping relation between parsing expression which allow us
to describe coercions between patterns to allow both an easy specification of tree
patterns and immediate implementation of pattern matching for trees.

\subsection{Pattern syntax and typing relation}

In simple terms, patterns describe the user intendeed structure for a given
input parse tree. Pattern syntax shares the same structure of tree syntax with
the addition of \emph{pattern-variables}, which will match any tree of a given type.
The syntax of pattern is presented next.

\[
    \begin{array}{lcl}
        p & \to & \epsilon \, \mid \, a \, \mid \, A@p\, \mid \,p_1\:p_2\,
                \mid\,p_1\,/\,p_2\, \mid \,p^\star\, \mid \,!\,p
                \mid \, M : e
    \end{array}
\]
Where \(\epsilon\) is a pattern that matches with the tree of
empty string (\(\hat{\epsilon}\)),
\(a\) matches only with the tree of the symbol \(a\) (\(\hat{a}\)), \(A\) matches
with a subtree of type \(e\) and \((A, e) \in R\), \(p_1\:p_2\) matches if both
\(p_1\) and \(p_2\) matches sequentially. \(p_1\,/\,p_2\) matches if one of \(p_1\)
or \(p_2\) matches, \(p^\star\) will try to match \(p\) sequentially as many times
as possible, \(!p\) matches only if \(p\) does not matches. \(M : e\) is a meta-variable
that matches with any tree \(t : e\).

Since patterns are similar to trees, we also define a typing relation for patterns.

\begin{figure}[H]
  \[
    \begin{array}{cc}
      \infer[_{\{PEps\}}]{G \vdash \epsilon : \epsilon}{} &
      \infer[_{\{PChr\}}]{G\vdash a : a}{} \\ \\
      \infer[_{\{PVar\}}]{ G \vdash A@p : A}
                      {A \leftarrow e \in R(G) & G\vdash p : e} &
      \infer[_{\{PNot\}}]{G\vdash !p: ! e}{G\vdash p : e}\\ \\
      \infer[_{\{PCat\}}]{G\vdash p_1, p_2 : e_1\,e_2}
                      {G\vdash  p_1 : e_1 & G\vdash p_2 : e_2} &
      \infer[_{\{PLeft\}}]{G\vdash p_1 : e_1 / e_2}
                       {G\vdash p_1 : e_1} \\ \\
      \infer[_{\{PRight\}}]{G\vdash p_2 : e_1 / e_2}
                       {G\vdash p_2 : e_2} &
      \infer[_{\{PStar\}}]{G \vdash  p^ * : e^*}{G \vdash p : e}\\ \\
   \end{array}
  \]
  \centering
  \caption{Typing relation for patterns.}
  \label{fig:pattern-typing}
\end{figure}

\subsection{Subtyping and for patterns}

When specifying patterns over choice expressions ... EXPLICAR o motivo do subtyping

We let notation $e <: e'$ denote that expression $e$ is a subtype of expression $e'$.
The idea of the subtyping relation is to express that is possible to ``coerce'' a
tree $t$, such that $G \vdash t : e$,  to a tree $t'$, such that $G \vdash t' : e'$,
in a way that the coercion preserves the parsed string, i.e., $|t| = |t'|$.
Since patterns shares all the structure of parse trees, the coercion reasoning
applies to patterns. The main objective of this section is to define, in a precise
manner, the subtype-based coercion for patterns.

We start by defining the subtype relation between parsing expressions.

\begin{figure*}[ht]
    \[
        \begin{array}{cccc}
            \infer[_{\{Refl\}}]{e <: e}{} &
            \infer[_{\{Alt_{Left}\}}]{e_1 <: e_2 / e_3}{e_1 <: e_2} &
            \infer[_{\{Alt_{Right}\}}]{e_1 <: e_2 / e_3}{e_1 <: e_3} &
            \infer[_{\{Star\}}]{e_1^n <: e_2^\star}{n \geq 0 & e_1 <: e_2}
        \end{array}
    \]
    \centering
    \caption{Subtype relation for parsing expressions}
    \label{fig:subtype-relation}
\end{figure*}

The first rule denote a common requirement for subtyping: it is
a reflexive relation. Next, we have two rules for dealing
with the choice operator. The first shows that the left component of a choice
is a subtype of $e_1/e_2$. The same holds for the right operand of ordered choice
expression. Finally, we have that $e_1^n$ is a subtype of $e_2^\star$, whenever
$e_1 <: e_2$. Notation $e^n$ denotes the repeated concatenation of $e$, as usual.

Next, we define the syntax of terms to denote subtyping proofs, as follows:

\[
  \begin{array}{lcl}
    d & \to & \mathbf{refl}
      \: \mid \: \mathbf{Inl}\,d
      \: \mid \: \mathbf{Inr}\,d
      \: \mid \: \mathbf{Rec}\,d
  \end{array}
\]
Term $\mathbf{refl}$ denotes a proof for $e<:e$ and $\mathbf{Inl}\:d$ denotes
a subtyping derivation for $e_1 <:e_2 / e_3$, where $d$ represents the
subderivation for $e_1 <: e_2$. Same reasoning applies to $\mathbf{Inr}$ and
$\mathbf{Rec}$ for deductions of $e_1 <: e_2 / e_3$ and $e_1^n <: e_2^*$,
respectively.

Finally, we show that the subtyping relation for parsing expressions is
decidable.

\begin{theorem}[Decidability of subtyping]
  For any parsing expressions $e_1$ and $e_2$, we have that either $e_1 <: e_2$ or
  that $\neg (e_1 <: e_2)$
\end{theorem}
\begin{proof}
  Induction over the structure of $e_1$ and case analysis over the structure of $e_2$
  using the definition of the subtyping relation.
\end{proof}


\subsection{Subtype-based coercion for patterns}

In this section we define an algorithm for coercing patterns using the subtyping
relation between parsing expressions. The algorithm is defined by equations over
pattern trees and subtyping terms.

\[
  \begin{array}{lcl}
    \mathbf{coerce}(p,\mathbf{refl}) & = & p\\
    \mathbf{coerce}(p,\mathbf{Inl}\:d) & = & \mathbf{coerce}(p,d)\,/\,\bot_p\\
    \mathbf{coerce}(p,\mathbf{Inr}\:d) & = & \bot_p\,/\,\mathbf{coerce}(p,d)\\
    \mathbf{coerce}(p^n,\mathbf{Rec}\:d) & = & \mathbf{coerce}(p,d)^\star\\
  \end{array}
\]

Notation $\bot_p$ denotes the patterns which fails for any parse tree. Intuitively,
the coercion allow us to match trees for the choice or star operator by just specifying
the desired component, for ordered choice, or a specific number of matchings for the
star operator.

The coerce function satisfies the following property.

\begin{theorem}[Correctness of coerce]
  Let $G = (V,\Sigma, R, e_s)$ be an arbitrary PEG, $p$ a pattern such that
  $G\vdash p : e$ and that $d$ is a proof term for $e <: e'$, for some
  $e'$. Then, $G\vdash \mathbf{coerce}(p,d) : e'$.
\end{theorem}
\begin{proof}
  Induction over the structure of $G\vdash p : e$ and case analysis over
  the derivation of $e <: e'$ using the definitions of $\mathbf{coerce}$ and
  the typing relation for patterns.
\end{proof}

Now with the definition of subtyping-based coercion for patterns we are
ready to formally define an algorithm for pattern matching parse trees.

\subsection{A semantics for parse trees pattern matching}

We define the process of pattern matching by and inductively defined judgment
$G \vdash p : e ; t \leadsto S$, where $G$ denotes a PEG, $p$ a pattern for $e$,
$t$ a parse tree and $S$ is a finite mapping between pattern variable names and parse
trees which are matched these variables during the algorithm execution.

\begin{figure}[H]
  \[
    \begin{array}{cc}
      \infer[_{\{PPVar\}}]{G\vdash M_1 : e ; t_1 \leadsto [M_1 \mapsto t_1]}{G \vdash t_1 : e} &
      \infer[_{\{PEps\}}]{G \vdash \epsilon : \epsilon ; \hat{\epsilon} : \leadsto \bullet}{} \\ \\
      \infer[_{\{PChr\}}]{G\vdash a ; \hat{a} : a \leadsto \emptyset}{} &
      \infer[_{\{PVar\}}]{G\vdash A@p ; A@t : A \leadsto S}
                      {A \leftarrow e \in R(G) & G\vdash p ; t : e \leadsto S}\\ \\
      \multicolumn{2}{c}{
        \infer[_{\{PCat\}}]{G\vdash p_1\,p_2 ; \langle t_1, t_2 : e_1\,e_2 \rangle \leadsto S}
                        {G\vdash p_1 ; t_1 : e_1 \leadsto S_1 &
                         G\vdash S_1[p_2] ; t_2 : e_2 \leadsto S}
      } \\ \\
    \end{array}
  \]
  \centering
  \caption{Semantics for pattern matching parse trees.}
  \label{fig:pattern-semantics}
\end{figure}

% Let \(G = (V, \Sigma, R, e_s)\) be an arbitrary PEG, \(\Theta\) a finite set
% of identified patterns, \(U\) a finite set of variables, the meta-variable
% \(a \in \Sigma\) an arbitrary alphabet symbol, \(A \in V\) a variable and
% \(e\) a parsing expression.

% \begin{definition}[Identified pattern]
%     An identified pattern is a pair \((i, p)\) where \(i\) is an identifier and
%     \(p\) is a pattern.
% \end{definition}
%
% Let \(\Theta\) a finite set of identified patterns, \(U\) a finite set of
% variables and \(A \in V\) a variable.
% Each identified pattern \(p_i \in \Theta\) is a pair \((I, p)\), where \(I \in U\)
% and \(p\) is a pattern.
% The following context-free grammar defines the syntax of a pattern:
% \[
%     \begin{array}{lcl}
%         p & \to & \epsilon \, \mid \, a \, \mid \, A\, \mid \,p_1\:p_2\,
%                 \mid\,p_1\,/\,p_2\, \mid \,p^\star\, \mid \,!\,p
%                 \mid \, M\, \mid \, I\,\\
%     \end{array}
% \]
% Where \(\epsilon\) is a pattern that matches with the tree of empty string (\(\hat{\epsilon}\)),
% \(a\) matches only with the tree of the symbol \(a\) (\(\hat{a}\)), \(A\) matches
% with a subtree of type \(e\) and \((A, e) \in R\), \(p_1\:p_2\) matches if both
% \(p_1\) and \(p_2\) matches sequentially. \(p_1\,/\,p_2\) matches if one of \(p_1\)
% or \(p_2\) matches, \(p^\star\) will try to match \(p\) sequentially as many times
% as possible, \(!p\) matches only if \(p\) does not matches. \(M\) is a meta-variable
% that, given a variable \(A\), matches with any tree \(t : e\) where \((A, e) \in R\)
% and \(I\) is a reference to another pattern \(p'\) where \((I, p') \in \Theta\).
%
% Figure~\ref{fig:pattern-semantics} defines the pattern semantics.
%
% \begin{figure*}[!ht]
%     \[
%         \begin{array}{ccc}
%             \infer[_{\{Eps\}}]{\Theta, G \vdash \epsilon}{} &
%             \infer[_{\{ChrS\}}]{\Theta, G \vdash a}{} &
%             \infer[_{\{Var\}}]{\Theta, G \vdash A}{A \in V}
%             \\ \\
%             \infer[_{\{Sequence\}}]{\Theta, G \vdash p_1\:p_2}{\Theta, G \vdash p_1 & \Theta, G \vdash p_2} &
%             \infer[_{\{Choice\}}]{\Theta, G \vdash p_1 \,/\, p_2}{\Theta, G \vdash p_1 & \Theta, G \vdash p_2} &
%             \infer[_{\{Star\}}]{\Theta, G \vdash p^\star}{\Theta, G \vdash p}
%             \\ \\
%             \infer[_{\{Not\}}]{\Theta, G \vdash !p}{\Theta, G \vdash p} &
%             \infer[_{\{MetaVar\}}]{\Theta, G \vdash M}{\exists e. M : e \land A \leftarrow e \in R} &
%             \infer[_{\{Ref\}}]{\Theta, G \vdash I}{\exists e. \Theta(I) = e}
%         \end{array}
%     \]
%     \centering
%     \caption{Patterns semantics.}
%     \label{fig:pattern-semantics}
% \end{figure*}
%
% \begin{definition}[Valid pattern with respect to a tree]
%     We say that a pattern \(p\) is valid with respect to a tree \(t\), \(p \sim t\),
%     if and only if \(\exists e . p : e \land t : e\).
% \end{definition}
%
% We also present a type coercion for parsing expressions.
% \begin{figure*}[ht]
%     \[
%         \begin{array}{ccc}
%             \infer[_{\{Reflexive\}}]{e <: e}{} &
%             \multicolumn{2}{c}{
%                 \infer[_{\{Transitive\}}]{e_1 <: e_3}{e_1 <: e_2 & e_2 <: e_3}
%             } \\ \\
%
%             \infer[_{\{Alt_{Left}\}}]{e_1 <: e_1 / e_2}{} &
%             \infer[_{\{Alt_{Right}\}}]{e_2 <: e_1 / e_2}{} &
%             \infer[_{\{Star\}}]{e^n <: e^\star}{n \geq 1}
%         \end{array}
%     \]
%     \centering
%     \caption{Subtype relations for parsing expressions}
%     \label{fig:subtype-relations}
% \end{figure*}
%
% We present the syntax for terms of subtyping.
% \[
%    \begin{array}{lcl}
%       p & \to & \epsilon \, \mid \, a \, \mid \, A \,
%                     \mid \,  p_1\:p_2 \, \mid \,  p_1 / p_2 \,
%                     \mid \, L\:p \, \mid \, R\:p \,
%                     \mid \, p^\star \, \mid \, [p] \,
%                     \mid \, !\,p \, \mid \, M \\
%    \end{array}
% \]
% Of note, are the production rules \(L\:p\), \(R\:p\) and \([p]\) which represents,
% respectively, the proof that the left expression in a choice operator is a subtype
% of the choice, the proof that the right expression in a choice operator is a subtype
% of the choice, and the proof that a list (possibly empty) of patterns is a subtype
% of the \(\star\) operator.
%
% \begin{figure*}[ht]
%     \[
%         \begin{array}{ccc}
%             \multicolumn{3}{c}{
%                 \infer[_{Pattern}]
%                     {p' \sim t}
%                     {p : e & e <: e' & \exists p' . p' = C(p, e <: e') & \forall t. t:e'}
%             }
%         \end{array}
%     \]
%     \centering
%     \caption{Pattern coercion}
%     \label{fig:pattern-coercion}
% \end{figure*}
% Where \(C\) is a resursively defined function that receives a pattern and a proof
% that the pattern is valid and returns a corrected pattern and is defined as follows:
% \[
%     \begin{array}{ll}
%         C(\epsilon, \epsilon)             & = \epsilon \\
%         C(a, a)                           & = a \\
%         C(a, a')                          & = \bot\text{, if } a \neq a' \\
%         C((A\:p), (A\:p'))                & = A C(p, p') \\
%         C((A\:p), (A'\:p'))               & = \bot\text{, if } A \neq A' \\
%         C(M, M)                           & = M \\
%         C(M, M')                          & = \bot\text{, if } M \neq M' \\
%         C(p_1\:p_2, p_1'\:p_2')           & = C(p_1, p_1')\:C(p_2, p_2') \\
%         C(\epsilon, xs))                  & = [] \\
%         C(p, [x])                         & = [C(p, x)] \\
%         C(p_1\:p_2, x:xs)                 & = C(p_1, x) \: C(p_2, xs) \\
%         C(p_1 \,/\, p_2, p_1' \,/\, p_2') & = C(p_1, p_1') / C(p_2, p_2') \\
%         C(p, L p')                        & = C(p, p') / !\,\epsilon \\
%         C(p, R p')                        & = !\,\epsilon / C(p, p') \\
%         C(p^\star, {p'}^\star)            & = (C(p, p'))^\star \\
%         C(!\,p, !\,p')                    & = !\,C(p, p') \\
%     \end{array}
% \]
%
%
% \textbf{-- Escrever regras de casamento --}
% \begin{figure*}[ht]
%     \[
%         \begin{array}{ccc}
%             % \infer[_{\{Eps\}}]{\epsilon \sim \epsilon}{} &
%
%             % \infer[_{\{ChrS\}}]{a \sim a}{} &
%
%             % \infer[_{\{Var\}}]{A \sim A}{} \\ \\
%
%             % \infer[_{\{Seq\}}]
%             %     {p_1\:p_2 \sim t_1\:t_2}
%             %     {p_1 \sim t_1 & p_2 \sim t_2} &
%
%             % \infer[_{\{Choice\}}]
%             %     {p_1 / p_2 \sim t_1 / t_2}
%             %     {p_1 \sim t_1 & p_2 \sim t_2} &
%
%             % \infer[_{\{Star\}}]
%             %     {p^\star \sim [t]}
%             %     {p \sim t} \\ \\
%
%             % \infer[_{\{Not\}}]
%             %     {!\,p \sim !\,t}
%             %     {p \sim t} &
%
%             % \infer[_{\{MetaVar\}}]{M ~ t}{\exists e. M:e \land t:e} &
%
%             % % Não tem definição de casamento pra uma referência,
%             % % pq ela não existe durante o casamento
%             % \\
%         \end{array}
%     \]
%     \centering
%     \caption{Matching rules}
%     \label{fig:matching-rules}
% \end{figure*}
%
% \section{Implementation details}\label{sec:implementation-details}

% After parsing the patterns, we replace references to other patterns with the
% pattern itself. To do this, we create a dependency graph between the patterns,
% topologically sort and replace the references so that no resulting pattern
% contains references to other patterns and can be treated as a single pattern.
%
% We also define two new operators for PEGs: flatten \((^\wedge e)\) and indentation
% \((e_1 > e_2)\). The former flattens the parsed tree in one single node that turns
% into a terminal (and can be matched as such via patterns) and the latter, for the
% purpose of parsing, acts as the sequence \(e_1\:{e_2}^\star\) with the restriction
% that \({e_2}^\star\) must be indented with respect to \(e_1\) and matches as if it
% was a normal sequence.
%

\section{Conclusion}\label{sec:methodology-conclusion}

---

\cleardoublepage
