\begin{abstract}

Casamento de padrões é um mecanismo usado em algumas linguagens de programação 
como uma ferramenta para processar dados com base em sua estrutura, e muitos 
editores de texto oferecem suporte à busca por expressões regulares. Programadores 
frequentemente utilizam ferramentas com suporte à casamento de padrões para tentar 
entender algum software, ou seja, para realizar uma análise estática co código.
Neste trabalho, apresentamos uma formalização para a produção de uma árvore de 
análise sintática ao executar uma \textit{Parsing Expression Grammar} arbitrária, 
uma relação de tipagem (e subtipagem) e um algoritmo de casamento de padrões sobre 
essas árvores de análise sintática, e uma ferramenta que implementa a formalização. 
A ferramenta foi originalmente projetada como uma forma de auxiliar um juiz automático
na avaliação da estrutura de um código submetido por um aluno, além de avaliar 
a saída produzida pelo código. Também apresentamos alguns estudos de caso para 
avaliar as capacidades da ferramenta.

\end{abstract}

%\newpage\null\thispagestyle{empty}\newpage