\chapter{Methodology}\label{chap:methodology}

This chapter presents the formalisms developed during the work, as well as the 
current state of the developed library. Section~\ref{sec:parse-tress} details
the execution of a PEG for generating a parse tree and how pattern matching works.
Section~\ref{sec:implementation-details} discusses some implementation details.
Section~\ref{sec:methodology-conclusion} concludes the chapter.

\section{Parse trees}\label{sec:parse-tress}

Let \(G = (V, \Sigma, R, e_s)\) be an arbitrary PEG, the meta-variable \(a \in \Sigma\) an
arbitrary alphabet symbol, \(A \in V\) a variable and \(e\) a parsing expression.
The following context-free grammar defines the syntax of a parse tree:
\[
   \begin{array}{lcl}
      t & \to & \hat{\epsilon} \, \mid \, \hat{a} \, \mid \, \hat{A} \, 
                    \mid \, \langle t_1, t_2 \rangle\,
                    \mid \, L \: t \, \mid \, R \: t \, \mid \, [t] \, 
                    \mid \, \eta \\
   \end{array}
\]
Where \(\hat{\epsilon}\) represents that a parsing expression resulted in 
success without consuming any symbol of its input, \(\hat{a}\) represents that
the parsing expression consumed the symbol \(a\) from the input, \(\hat{A}\)
represents that the parsing of the rule \((A, e) \in R\) was succeesful, 
\(\langle t_1, t_2 \rangle\) represents that a sequence of parsing expressions 
succeeded, 
\(L \: t\) and \(R \: t\) both represent that a branch of an ordered choice succeeded, 
with \(L \: t\) for the left one and \(R \: t\) for the right one,
% \(L \: t\) represents that the first expression in an ordered choice 
% succeeded, \(R \: t\) represents that the second expression in an ordered choice 
% succeeded, 
\([t]\) is a list of trees and \(\eta\) represents that a not 
predicate was successful.

Executing a parsing expressions produces a parsing tree and is defined by an 
inductively defined judgment that relates pairs formed by a parsing expression 
and an input string to pairs formed by the generated tree and the remaining string.
Notation \((e,s_ps_r) \Rightarrow_G (t,s_r)\) denote that parsing expression \(e\)
consumes the prefix \(s_p\) and generates the parse tree \(t\) from the input string
\(s_ps_r\) leaving the suffix \(s_r\). The notation \((e,s) \Rightarrow_G \bot\) 
denote the fact that \(s\) cannot be parsed by \(e\). We let meta-variable \(r\) 
denote an arbitrary parsing result, i.e., either \(r\) is a pair \((t,s_r)\) or 
\(\bot\). We say that an expression \(e\) fails if its execution over an input 
produces \(\bot\); otherwise, it succeeds. Figure~\ref{fig:peg-tree-semantics} 
defines the PEG semantics for tree generation.

\begin{figure*}[!h]
   \[
      \begin{array}{cccc}
         \infer[_{\{Eps\}}]{(\epsilon,s) \Rightarrow_G (\hat{\epsilon},s)}{} &
         \infer[_{\{ChrS\}}]{(a,as_r) \Rightarrow_G (\hat{a},s_r)}{}  &
         \infer[_{\{ChrF\}}]{(a,bs_r) \Rightarrow_G \bot}{a \neq b} &
         \infer[_{\{Var\}}]{(A,s) \Rightarrow_G t}
                           {A \leftarrow e \in R & (e,s) \Rightarrow_G t} \\ \\
         \multicolumn{2}{c}{
            \infer[_{\{Cat_{S1}\}}]{(e_1\,e_2,s_{p_1}s_{p_2}s_r) \Rightarrow_G (\langle t_1,t_2 \rangle, s_r)}
                                 {(e_1,s_{p_1}s_{p_2}s_r) \Rightarrow_G (t_1,s_{p_2}s_r) &
                                 (e_2,s_{p_2}s_r)\Rightarrow_G (t_2,s_r)}
         } &
         \multicolumn{2}{c}{
            \infer[_{\{Cat_{F2}\}}]{(e_1\,e_2,s_ps_r) \Rightarrow_G \bot}
                                 { (e_1,s_ps_r) \Rightarrow_G (t_1,s_r) &
                                    (e_2,s_r) \Rightarrow_G \bot}} \\ \\
         \infer[_{\{Cat_{F1}\}}]{(e_1\,e_2,s)\Rightarrow_G \bot}{(e_1,s) \Rightarrow_G \bot} &
         \infer[_{\{Alt_{S1}\}}]{(e_1\,/\,e_2,s_p\,s_r) \Rightarrow_G (L \, t,s_r)}
                                {(e_1,s_p\,s_r)\Rightarrow_G (t,s_r)} &
         \multicolumn{2}{c}{
            \infer[_{\{Alt_{S2}\}}]{(e_1\,/\,e_2,s_p\,s_r) \Rightarrow_G R \, t}
                                  {(e_1,s_p\,s_r)\Rightarrow_G \bot &
                                   (e_2,s_p\,s_r)\Rightarrow_G t}
         } \\ \\
         \multicolumn{2}{c}{
            \infer[_{\{Star_{rec}\}}]{(e^\star,s_{p_1}s_{p_2}s_r) \Rightarrow_G ([t_1, t_2],s_r)}
                                 {(e,s_{p_1}s_{p_2}s_r) \Rightarrow_G (t_1,s_{p_2}s_r) &
                                  (e^\star, s_{p_2}s_r) \Rightarrow_G (t_2,s_r)}
         } &
         \multicolumn{2}{c}{
            \infer[_{\{Star_{end}\}}]{(e^\star,s) \Rightarrow_G (\hat{\epsilon},s)}
                                    {(e,s) \Rightarrow_G \bot}} \\ \\
         \multicolumn{2}{c}{
            \infer[_{\{Not_F\}}]{(!\,e,s_p\,s_r) \Rightarrow_G \bot}
                               {(e,s_p\,s_r) \Rightarrow_G (t,s_r)}
         } &
         \infer[_{\{Not_S\}}]{(!\,e,s) \Rightarrow_G (\eta),s)}
         {(e,s) \Rightarrow_G \bot}
           &
         \infer[_{\{ChrNil\}}]{(a,\epsilon) \Rightarrow_G \bot}{}
      \end{array}
   \]
   \centering
   \caption{Parsing expressions operational semantics that produces a tree.}
   \label{fig:peg-tree-semantics}
\end{figure*}

\begin{definition}[Type of a parse tree]
    We say that a parse tree \(t\) has type \(e\), \(t : e\), when \(t\) is generated 
    by a parsing expression \(e\), i.e., when \((e, s_ps_r) \Rightarrow_G (t, s_r)\).
\end{definition}


% Let \(G = (V, \Sigma, R, e_s)\) be an arbitrary PEG, \(\Theta\) a finite set
% of identified patterns, \(U\) a finite set of variables, the meta-variable 
% \(a \in \Sigma\) an arbitrary alphabet symbol, \(A \in V\) a variable and 
% \(e\) a parsing expression.

\begin{definition}[Identified pattern]
    An identified pattern is a pair \((i, p)\) where \(i\) is an identifier and
    \(p\) is a pattern.
\end{definition}

Let \(\Theta\) a finite set of identified patterns, \(U\) a finite set of 
variables and \(A \in V\) a variable.
Each identified pattern \(p_i \in \Theta\) is a pair \((I, p)\), where \(I \in U\)
and \(p\) is a pattern.
The following context-free grammar defines the syntax of a pattern:
\[
    \begin{array}{lcl}
        p & \to & \epsilon \, \mid \, a \, \mid \, A\, \mid \,p_1\:p_2\,
                \mid\,p_1\,/\,p_2\, \mid \,p^\star\, \mid \,!\,p 
                \mid \, M\, \mid \, I\,\\
    \end{array}
\]
Where \(\epsilon\) is a pattern that matches with the tree of empty string (\(\hat{\epsilon}\)), 
\(a\) matches only with the tree of the symbol \(a\) (\(\hat{a}\)), \(A\) matches 
with a subtree of type \(e\) and \((A, e) \in R\), \(p_1\:p_2\) matches if both 
\(p_1\) and \(p_2\) matches sequentially. \(p_1\,/\,p_2\) matches if one of \(p_1\) 
or \(p_2\) matches, \(p^\star\) will try to match \(p\) sequentially as many times 
as possible, \(!p\) matches only if \(p\) does not matches. \(M\) is a meta-variable 
that, given a variable \(A\), matches with any tree \(t : e\) where \((A, e) \in R\) 
and \(I\) is a reference to another pattern \(p'\) where \((I, p') \in \Theta\).

Figure~\ref{fig:pattern-semantics} defines the pattern semantics.

\begin{figure*}[!ht]
    \[
        \begin{array}{ccc}
            \infer[_{\{Eps\}}]{\Theta, G \vdash \epsilon}{} &
            \infer[_{\{ChrS\}}]{\Theta, G \vdash a}{} &
            \infer[_{\{Var\}}]{\Theta, G \vdash A}{A \in V}
            \\ \\
            \infer[_{\{Sequence\}}]{\Theta, G \vdash p_1\:p_2}{\Theta, G \vdash p_1 & \Theta, G \vdash p_2} &
            \infer[_{\{Choice\}}]{\Theta, G \vdash p_1 \,/\, p_2}{\Theta, G \vdash p_1 & \Theta, G \vdash p_2} &
            \infer[_{\{Star\}}]{\Theta, G \vdash p^\star}{\Theta, G \vdash p} 
            \\ \\
            \infer[_{\{Not\}}]{\Theta, G \vdash !p}{\Theta, G \vdash p} &
            \infer[_{\{MetaVar\}}]{\Theta, G \vdash M}{\exists e. M : e \land A \leftarrow e \in R} &
            \infer[_{\{Ref\}}]{\Theta, G \vdash I}{\exists e. \Theta(I) = e}
        \end{array}
    \]
    \centering
    \caption{Patterns semantics.}
    \label{fig:pattern-semantics}
\end{figure*}

\begin{definition}[Valid pattern with respect to a tree]
    We say that a pattern \(p\) is valid with respect to a tree \(t\), \(p \sim t\),
    if and only if \(\exists e . p : e \land t : e\).
\end{definition}

We also present a type coercion for parsing expressions.
\begin{figure*}[ht]
    \[
        \begin{array}{ccc}
            \infer[_{\{Reflexive\}}]{e <: e}{} &
            \multicolumn{2}{c}{
                \infer[_{\{Transitive\}}]{e_1 <: e_3}{e_1 <: e_2 & e_2 <: e_3}
            } \\ \\

            \infer[_{\{Alt_{Left}\}}]{e_1 <: e_1 / e_2}{} &
            \infer[_{\{Alt_{Right}\}}]{e_2 <: e_1 / e_2}{} &
            \infer[_{\{Star\}}]{e^n <: e^\star}{n \geq 1} 
        \end{array}
    \]
    \centering
    \caption{Subtype relations for parsing expressions}
    \label{fig:subtype-relations}
\end{figure*}

We present the syntax for terms of subtyping.
\[
   \begin{array}{lcl}
      p & \to & \epsilon \, \mid \, a \, \mid \, A \, 
                    \mid \,  p_1\:p_2 \, \mid \,  p_1 / p_2 \,
                    \mid \, L\:p \, \mid \, R\:p \, 
                    \mid \, p^\star \, \mid \, [p] \, 
                    \mid \, !\,p \, \mid \, M \\
   \end{array}
\]
Of note, are the production rules \(L\:p\), \(R\:p\) and \([p]\) which represents,
respectively, the proof that the left expression in a choice operator is a subtype
of the choice, the proof that the right expression in a choice operator is a subtype
of the choice, and the proof that a list (possibly empty) of patterns is a subtype 
of the \(\star\) operator.

\begin{figure*}[ht]
    \[
        \begin{array}{ccc}
            \multicolumn{3}{c}{
                \infer[_{Pattern}]
                    {p' \sim t}
                    {p : e & e <: e' & \exists p' . p' = C(p, e <: e') & \forall t. t:e'}
            }
        \end{array}
    \]
    \centering
    \caption{Pattern coercion}
    \label{fig:pattern-coercion}
\end{figure*}
Where \(C\) is a resursively defined function that receives a pattern and a proof
that the pattern is valid and returns a corrected pattern and is defined as follows:
\[
    \begin{array}{ll}
        C(\epsilon, \epsilon)             & = \epsilon \\
        C(a, a)                           & = a \\
        C(a, a')                          & = \bot\text{, if } a \neq a' \\
        C((A\:p), (A\:p'))                & = A C(p, p') \\
        C((A\:p), (A'\:p'))               & = \bot\text{, if } A \neq A' \\
        C(M, M)                           & = M \\
        C(M, M')                          & = \bot\text{, if } M \neq M' \\
        C(p_1\:p_2, p_1'\:p_2')           & = C(p_1, p_1')\:C(p_2, p_2') \\
        C(\epsilon, xs))                  & = [] \\
        C(p, [x])                         & = [C(p, x)] \\
        C(p_1\:p_2, x:xs)                 & = C(p_1, x) \: C(p_2, xs) \\
        C(p_1 \,/\, p_2, p_1' \,/\, p_2') & = C(p_1, p_1') / C(p_2, p_2') \\
        C(p, L p')                        & = C(p, p') / !\,\epsilon \\
        C(p, R p')                        & = !\,\epsilon / C(p, p') \\
        C(p^\star, {p'}^\star)            & = (C(p, p'))^\star \\
        C(!\,p, !\,p')                    & = !\,C(p, p') \\
    \end{array}
\]


\textbf{-- Escrever regras de casamento --}
\begin{figure*}[ht]
    \[
        \begin{array}{ccc}
            % \infer[_{\{Eps\}}]{\epsilon \sim \epsilon}{} &

            % \infer[_{\{ChrS\}}]{a \sim a}{} &

            % \infer[_{\{Var\}}]{A \sim A}{} \\ \\

            % \infer[_{\{Seq\}}]
            %     {p_1\:p_2 \sim t_1\:t_2}
            %     {p_1 \sim t_1 & p_2 \sim t_2} &

            % \infer[_{\{Choice\}}]
            %     {p_1 / p_2 \sim t_1 / t_2}
            %     {p_1 \sim t_1 & p_2 \sim t_2} &

            % \infer[_{\{Star\}}]
            %     {p^\star \sim [t]}
            %     {p \sim t} \\ \\

            % \infer[_{\{Not\}}]
            %     {!\,p \sim !\,t}
            %     {p \sim t} &

            % \infer[_{\{MetaVar\}}]{M ~ t}{\exists e. M:e \land t:e} &

            % % Não tem definição de casamento pra uma referência, 
            % % pq ela não existe durante o casamento
            % \\
        \end{array}
    \]
    \centering
    \caption{Matching rules}
    \label{fig:matching-rules}
\end{figure*}

\section{Implementation details}\label{sec:implementation-details}

After parsing the patterns, we replace references to other patterns with the 
pattern itself. To do this, we create a dependency graph between the patterns, 
topologically sort and replace the references so that no resulting pattern 
contains references to other patterns and can be treated as a single pattern.

We also define two new operators for PEGs: flatten \((^\wedge e)\) and indentation 
\((e_1 > e_2)\). The former flattens the parsed tree in one single node that turns
into a terminal (and can be matched as such via patterns) and the latter, for the 
purpose of parsing, acts as the sequence \(e_1\:{e_2}^\star\) with the restriction 
that \({e_2}^\star\) must be indented with respect to \(e_1\) and matches as if it 
was a normal sequence.

\section{Conclusion}\label{sec:methodology-conclusion}

---

\cleardoublepage